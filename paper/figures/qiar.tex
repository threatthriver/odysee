\documentclass{standalone}
\usepackage{tikz}
\usetikzlibrary{quantum}
\usetikzlibrary{arrows,shapes,positioning,shadows,trees}

\begin{document}
\begin{tikzpicture}[
    node distance=2cm,
    box/.style={rectangle,draw,rounded corners,fill=blue!10,minimum width=2cm,minimum height=1cm},
    qgate/.style={rectangle,draw,fill=red!10,minimum width=1cm,minimum height=1cm},
    arrow/.style={->,>=stealth,thick}
]

% Input state
\node[box] (input) at (0,0) {Input State};

% Quantum circuit
\node[qgate] (h1) at (3,1) {$H$};
\node[qgate] (h2) at (3,0) {$H$};
\node[qgate] (h3) at (3,-1) {$H$};

\node[qgate] (r1) at (5,1) {$R_\phi$};
\node[qgate] (r2) at (5,0) {$R_\phi$};
\node[qgate] (r3) at (5,-1) {$R_\phi$};

\node[qgate] (cnot1) at (7,0.5) {CNOT};
\node[qgate] (cnot2) at (7,-0.5) {CNOT};

% Output state
\node[box] (output) at (10,0) {Routing Weights};

% Connections
\draw[arrow] (input) -- node[above] {Prepare} (h1);
\draw[arrow] (input) -- (h2);
\draw[arrow] (input) -- (h3);

\draw[arrow] (h1) -- node[above] {Phase} (r1);
\draw[arrow] (h2) -- (r2);
\draw[arrow] (h3) -- (r3);

\draw[arrow] (r1) -- node[above] {Entangle} (cnot1);
\draw[arrow] (r2) -- (cnot1);
\draw[arrow] (r2) -- (cnot2);
\draw[arrow] (r3) -- (cnot2);

\draw[arrow] (cnot1) -- node[above] {Measure} (output);
\draw[arrow] (cnot2) -- (output);

% Labels
\node[above=0.5cm of input] {Quantum-Inspired Adaptive Routing};

\end{tikzpicture}
\end{document}
